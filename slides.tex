%----------------------------------------------------------------------------------------
%	PACKAGES AND THEMES
%----------------------------------------------------------------------------------------
\documentclass[aspectratio=169,xcolor=dvipsnames,german]{beamer}
\usepackage{amsmath} % math package
\usepackage{minted} % include code snippets
\usepackage[utf8]{inputenc}
\usepackage[main=german, english]{babel} % translations
\usepackage{csquotes} % Better quoting
\usepackage[
    backend=biber,
    style=alphabetic,
    sorting=ynt
]{biblatex}
\addbibresource{bibliography.bib}
\usetheme{Madrid}
\usecolortheme{beaver}
\usepackage{hyperref}
\usepackage{graphicx} % Allows including images
\usepackage{booktabs} % Allows the use of \toprule, \midrule and \bottomrule in tables

%----------------------------------------------------------------------------------------
%	TITLE PAGE
%----------------------------------------------------------------------------------------

% The title
\title[Titel]{Titel}
\logo{\includegraphics[scale=0.25]{img/DHBWLogo.png}}
\subtitle{Untertitel}

\author[Vorname Name] {Vorname Name}
\institute[DHBW Mannheim] % Your institution may be shorthand to save space
{
    % Your institution for the title page
    Duale Hochschule Baden Württemberg Mannheim
    \vskip 3pt
}
\date{\today} % Date, can be changed to a custom date


%----------------------------------------------------------------------------------------
%	PRESENTATION SLIDES
%----------------------------------------------------------------------------------------

\begin{document}

\begin{frame}
    % Print the title page as the first slide
    \titlepage
\end{frame}

%----------------------------------------------------------------------------------------

\begin{frame}{Inhaltsverzeichnis}
    % Throughout your presentation, if you choose to use \section{} and \subsection{} commands, these will automatically be printed on this slide as an overview of your presentation
    \tableofcontents
\end{frame}

%----------------------------------------------------------------------------------------
% CONTENT SLIDES
%----------------------------------------------------------------------------------------

\begin{frame}{Blocks of Highlighted Text}
    Highlighting per \alert{\textbackslash alert}

    \begin{block}{Block}
        Lorem ipsum dolor sit amet
    \end{block}

    \begin{alertblock}{Alertblock}
        Lorem ipsum dolor sit amet
    \end{alertblock}

    \begin{examples}
        Lorem ipsum dolor sit amet
    \end{examples}
\end{frame}

%------------------------------------------------

\begin{frame}{Mehrere Spalten}
    \begin{columns}[c] % The "c" option specifies centered vertical alignment while the "t" option is used for top vertical alignment

        \column{.45\textwidth} % Left column and width
        \textbf{Überschrift}
        \begin{enumerate}
            \item Punkt
            \item Punkt
            \item Punkt
        \end{enumerate}

        \column{.5\textwidth} % Right column and width
        Lorem ipsum dolor sit amet, consectetur adipiscing elit. Integer lectus nisl, ultricies in feugiat rutrum, porttitor sit amet augue. Aliquam ut tortor mauris. Sed volutpat ante purus, quis accumsan dolor.

    \end{columns}
\end{frame}

%------------------------------------------------
\section{Mein erstes Dokument}
%------------------------------------------------

\begin{frame}{Erstellen eines neuen Dokuments}
\end{frame}

%------------------------------------------------

\begin{frame}{Theorem}
\end{frame}

%------------------------------------------------

\begin{frame}{Grafik}
    %\begin{figure}
    %\includegraphics[width=0.8\linewidth]{test}
    %\end{figure}
\end{frame}

%------------------------------------------------

\begin{frame}[fragile] % Need to use the fragile option when verbatim is used in the slide
    % \frametitle{Citation}
    % An example of the \verb|\cite| command to cite within the presentation:\\~

    % This statement requires citation \cite{p1}.
\end{frame}

%------------------------------------------------

\begin{frame}{Referenzen}
    % Beamer does not support BibTeX so references must be inserted manually as below
    % \footnotesize{
    %     \begin{thebibliography}{99}
    %         \bibitem[Smith, 2012]{p1} John Smith (2012)
    %         \newblock Title of the publication
    %         \newblock \emph{Journal Name} 12(3), 45 -- 678.
    %     \end{thebibliography}
    % }
\end{frame}

%------------------------------------------------
% SOURCES
%------------------------------------------------

\begin{frame}{Quellen}
    \printbibliography
\end{frame}

%------------------------------------------------
% WRAP UP
%------------------------------------------------

\begin{frame}
    \Huge{\centerline{Danke für Ihre Aufmerksamkeit!}}
\end{frame}

%----------------------------------------------------------------------------------------

\end{document}